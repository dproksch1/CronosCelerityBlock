\section{Output}
All output gets written into the {\it datadir} directory, i.e.\ the
same directory in which the cat file resides. {\sc Cronos} will create the
following four types of output files at the respective time intervals
specified in the cat file:

\subsection{HDF double}
\label{sec:out_double}
Name pattern:
$pname${\tt\_double}/$pname${\tt\_dbl\_step$\left<n\right>$.h5} \\
where $\left<n\right>$ is the integer running number of the timestep
at which the output occurred. HDF double files contain the complete
data for all variables on the entire grid with {\tt double} precision.\\
Purpose: provide data to restart a previously aborted or finished
run at some intermediate time.\\
If more than one processor is used and the \verb+CRONOS_OUTPUT_COMPATIBILTY+
flag is enabled, each task will either generate its own output
file, all of which then have their filenames extended by
\verb+_coord+$c_x${\tt-}$c_y${\tt-}$c_z$, where
the integer \mbox{$c_{x,y,z} \in \{0,\ {\tt nproc\{x,y,z\}} \}$} designate the
position of the task's sub-grid.

\subsection{HDF float}
\label{sec:out_float}
Name pattern:
$pname${\tt\_float}/$pname${\tt\_flt\_step$\left<n\right>$.h5} \\ Same
contents as HDF double files, except using single
precision.\\ Purpose: provide data for scientific analysis and
visualisation. By converting the double precision data to single
precision the memory demand is nearly cutting in half, while still
maintaining sufficiently high accuracy for data analysis. For
multi-processor runs, the same naming convention as for HDF double
files applies. Separate files from each process may be merged into one
single file (per timestep) using the deprecated {\em Converter} tool
(see Sect.~\ref{s:converter}) if so desired.

\subsection{HDF grid}
\label{sec:out_grid}

Name pattern:
$pname${\tt\_grid}/$pname${\tt\_grid.h5} \\
HDF file containing the Cartesian location of every grid point (cell center),
along with the transformation matrix with which the curvilinear coordinates
can be reconstructed.\\
Purpose: Needed for data processing using e.g.\ {\em Paraview} (see
Sect.~\ref{s:paraview}).
This type of output occurs only once at the very beginning of a run, and has
to be explicitly requested by providing the keyword {\tt WriteGrid}
(without arguments) in the cat file. Even for MPI runs, only one single file
containing the entire grid will be generated.

The directory $pname${\tt\_grid} also holds files describing the
general grid setup for all subprocesses of a simulation. For a
single-core run this information is stored in
$pname${\tt\_grid.info}. For MPI runs the naming scheme is
$pname${\tt\_grid.info}$Rank$, where $Rank$ is the number of the
individual sub-process.

\subsection{HDF mov}
Name pattern:
$pname${\tt\_mov}/$pname${\tt\_mov\_$qname$\_$plane$.h5}\\ where
$qname$ enumerates all integrated quantities (e.g.\ {\tt rho}, {\tt
  v\_x}, $\ldots$, one separate file for each of them), while $plane$
denotes the coordinate plane, which is one of {\tt xy}, {\tt xz}, or
{\tt yz}. This results in a total of \mbox{$3 \times [8(7)+{\tt
      N\_OMINT\_USER}]$} mov files for the entire simulation. For each
variable, its HDF mov files combine plane slices for all output times
into a single file, i.e.\ the spatial coordinate not mentioned in the
filename is replaced by a time axis.\\ Purpose: useful to analyse the
temporal evolution of a quantity, provided the planes of interest are
known beforehand. Due to the much smaller size (2D slices instead of
full 3D fields) movie files can be written more often than HDF double
or HDF float files. The standard setup writes cuts through the middle
of the numerical domain. A direct access to the position of this cut
is planned, but not implemented yet. Currently the only option is to
change the routine \verb+InitMovie+ in the file
\verb+generic/gridgen.C+.

\subsection{ASCII}
Name pattern: {\it pname}{\tt\_efluct}. Simple text file which
contains a table of some statistical values for each desired time
step. Can be read with any text editor, and processed using e.g.\
{\tt gnuplot}.
