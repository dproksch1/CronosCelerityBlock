\section{Further data processing}

Two more tools are available in the {\tt src/cronos/DataReader} directory, and
can be compiled and made available by {\tt cd}'ing into that directory and
executing {\tt make} .

\subsection{Merging HDF files from MPI runs}
\label{s:converter}

HDF float output files from MPI runs can be joined into one large file for the
whole grid using the {\em Converter} tool. After compilation, one then needs
to set the environment variables {\tt pname} and {\tt poub}
appropriately (relative to the directory from which the converter will be
called), and envoke {\tt DataReader/}{\em machinetype}{\tt /Converter} .
Then type {\tt 1} for hdf5 input, and select what timestep to convert.

\subsection{CronosPlot (IDL)}

(check in {\tt src/Visualization} ...)

\subsection{Paraview}
\label{s:paraview}

Paraview is a generic visualization package for large
three-dimensional datasets, and is freely available for various computer
platforms at \url{www.paraview.org}. In oder to load and view HDF float files
written
by {\sc Cronos} for a given project, an additional XML text file must be
created for that project to include additional information mandatory for the
correct interpretation of the heavy HDF data. This is done via the
{\sc XdmfGen} tool, which is also created with a simple {\tt make} in that
directory. Typing {\em machinetype}{\tt /XdmfGen} (without arguments) prints
information about the usage, which is just \\
\verb+  XdmfGen <poub> <pname> <itime_1> [<itime_2> [...] ]+ \\
where the {\em itime}s are numbers from the filenames of timesteps to include.
Alternatively, the script {\tt batch\_XdmfGen.sh} can be used to refer to the
complete set of all available timesteps.
As a second requirement, a \textit{pname}\verb+_grid.h5+ file must be present in the \textit{pname}\verb+_grid+ directory (see Sect.~\ref{sec:out_grid}).
After successful completion, the file {\em poub}/{\em pname}{\tt.xmf} can be
read directly into Paraview.
