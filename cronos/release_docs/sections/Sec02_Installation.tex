\section{Installation}


\begin{enumerate}
%\item Navigate to \url{https://git.uibk.ac.at/users/sign_in} and register by
%  klicking onto the field of that name, noting your chosen username and
%  password. Make sure you the login box says ``Standard'', not ``LDAP Server''.
\item Contact \href{mailto:ralf.kissmann@uibk.ac.at}{Ralf Kissmann} to obtain
  access to the Cronos repository at \url{https://git.uibk.ac.at}.%, quoting your user name.
%\item One you have been added to the project, return to the above URL to log in.
%  In the top-right corner, klick ``Profile settings,'' then ``SSH Keys''.
%  Past your SSH key into the ``Key'' box, and a suitable name into the
%  ``Title'' box below. If you do not yet have an SSH key, consult
%  \url{https://www.howtoforge.com/linux-basics-how-to-install-ssh-keys-on-the-shell} (or a similar ressource) on how to generate one.
\item In your shell, create your local working directory for Cronos
  (e.g. {\tt src}) and change into it using \verb+mkdir src && cd src+
\item \verb+git clone git@git.uibk.ac.at:c706138/CronosNumLib.git+ \\
  This will create a directory \mbox{\tt CronosNumLib} and several files
  therein. 
\item {\tt cd} into {\tt CronosNumlib} and follow steps listed in the
  {\tt README} file.
\item Move up into {\tt src} by {\tt cd ..} and check out Cronos code files
  with \\
  \verb+ git clone git@git.uibk.ac.at:c706138/CronosCode.git+ 
  If you have permission to \underline{write} files to the repository,
  you should also execute \verb+git push -u origin master+.
\item Follow steps described in {\tt CronosCode/README.md}.
  %  Note: \verb+CRONOS_ROOT_DIR+ is actually \verb+X_ROOT_DIR+.
  %(Example: \verb+CRONOS_ROOT_DIR = $(HOME)/src+)
  In our example case, \verb+'path-to-CronosNumLib'+ is
  \verb+'~/src/CronosNumLib'+. Do not use relative path specifiers!
\end{enumerate}

\ul{Note:} In order to compile properly on the TP4 cluster, the include and lib entries in file
\verb+CronosNumLib/makeinclude/include.mak+ should read

\begin{verbatim}
X_INC = -I$(X_LIB_ROOT_DIR)/include -I/usr/lib/x86_64-linux-gnu/openmpi/include \
                                    -I/usr/include/hdf5/openmpi
X_LIB = -L$(X_LIB_ROOT_DIR)/lib/$(X_OSTYPE) -L/usr/lib/x86_64-linux-gnu/hdf5/openmpi
\end{verbatim}

% Note: The installation procedure will create and populate directories
% {\tt include}, {\tt lib}, and {\tt src} in your home
% directory. Pre-existing files inside them may get overwritten. Be sure
% to use backups where appropriate! 
% \begin{enumerate}
% \item To download the source code, you will need an account at
%   TP4. Open a shell and type \\ {\tt
%     svn checkout \textbackslash \\
%     http://astro-staff.uibk.ac.at/svn-repo/cronos\_code \textbackslash \\
%     --username=cronos\_user} .\\
%   You will be prompted for a password, which may be obtained from either Ralf
%   or Jens. Upon checkout,
%   directories {\tt src/cronos}, {\tt src/makeinclude}, {\tt src/Matrix},
%   {\tt src/util}, and {\tt src/Visualisation} will be created, with
%   subdirectories and source files therein. Note that all pre-existing files
%   of the same name will be overwritten without prior warning!
% \item Inspect the file {\tt src/makeinclude/}{\it machinetype}, where
%   {\it machinetype} (e.g. {\tt Linux-amd64}) refers to the designation of
%   your machine/operating system. If unsure, consult the output of
%   \mbox{\tt uname -a}. In this file, make sure that the compiler flags and
%   paths are set up correctly, and make changes where neccessary.
%   For the Opteron cluster at TP4, {\tt MPICXX} must be set to
%   {\tt /opt/mpichgm-1.2.7p1..16/bin/mpiCC} .
% \item Then do:\\
%   {\tt mkdir include lib}\\
%   if these directories do not already exist. On a 64bit machine, type
%   {\tt mkdir lib/}{\it machinetype} . Then
%   {\tt cd} into {\tt src/util} and type {\tt make \&\& make
%     install}.  Repeat for {\tt src/Matrix}. At this point, the two
%   libraries {\tt libmatrix\_mt.a} and {\tt libutil\_mt.a} should be
%   present in your {\tt lib/}{\it machinetype} directory.
% \item Obtain {\tt hdf5-1.8.5.tar} from
%   {\tt http://www.hdfgroup.org/ftp/HDF5/prev-releases} \\
%   and copy it into your home directory. From there, execute the following:
% \begin{verbatim}
% tar -xvf hdf5-1.8.5.tar
% mkdir src/numlib
% mv hdf5-1.8.5 src/numlib/HDF
% cd src/numlib/HDF

% ./configure --enable-cxx
% make
% make check
% make install
% make check-install
% \end{verbatim}
%   This will take some time.
%   Upon successful completion, finish installation by \\
%   \verb+mv hdf5/include/* ~/include/+ \\
%   \verb+mv hdf5/lib/* ~/lib/+{\it machinetype}
% \end{enumerate}

See Section~\ref{sec:trouble_ahead} for additional installation
troubleshooting.


\subsection{Parallel HDF output}
To enable parallel output of all involved processors to a single HDF5 file
during MPI runs, the following steps have to be taken:
\begin{enumerate}
\item Install the respective hdf5 package {\tt libhdf5-openmpi-1.8.4}
  (for Debian and Ubuntu). At TP4, this is pre-installed on all 8-core
  machines and Intel clusters.
\item Make the following changes in {\tt src/makeinclude} :
  \begin{enumerate}
  \item In {\tt include.mak}, change
    \verb+X_INC = -I\$(X_ROOT_DIR)/include+ to \\
    \verb+X_INC = -I\$(X_ROOT_DIR)/include -I/usr/lib/openmpi/include+. 
  \item In the file for the respective operating system (e.g.\
    {\tt Linux-amd64}) use the following setting for CXX:
    {\tt CXX = g++ -DOMPI\_SKIP\_MPICXX} \\
    These changes are necessary to maintain the code's capability to produce
    output for non-MPI simulations.
  \end{enumerate}
\item Create backups of your {\tt lib} and {\tt include} folders located in
  your home folder. Remove all HDF-related files in {\tt lib} and
  {\tt include}. If you want to go back to serial output for each processor,
  you will have to restore the backups of these folders and undo the change
  to \\ {\tt cronos/generic/Hdf5file\_cbase.H} (next step).
\item Set {\tt \#define HDF\_PARALLEL\_IO SWITCH\_ON} in \\
  {\tt cronos/generic/Hdf5file\_cbase.H}.
\item In your call to {\tt mpirun} omit the option for the machine file
  ({\tt -mf} ...). A bash script that takes care of all this can be found
  at {\tt /home/home1/tow/bash-script/cronos-mpi.sh}
\end{enumerate}
{\bf NOTE:} The time it takes to write an HDF file in this manner is
slightly higher than in the case of multiple files for each processor. In
case of rapid output it can be advantageous to use the serial output.
However, the advantage of the parallel output is not to have to take care of
merging the serial files into one file, which is time consuming also and
necessary for data analysis. Also, the parallel output can be easily
analysed already during runtime.  



\subsection{Working with git}
The main {\tt CronosCode} repository comes with three different branches corresponding to different
development states of the code. The user can switch between them by using
{\tt git checkout <branch-name>}.\\
Starting from the bottom, the {\bf master-alpha} branch contains the bleeding edge development
versions of Cronos.
This branch can be updated quite frequently depending on the developer's activity.
It should not be used for production, since it can contain work in progress and is certainly prone
to contain bugs. After finishing a development step an automated testing of the code will be
performed. If the code passes all setups the commit receives a new version tag (e.g. {\tt 0.1.x})
and is merged to the {\bf master-beta} branch. This will happen on the timescale of weeks,
depending on the development activity.
Hotfixes, depending on their severity, can be pushed directly to this branch.
Cronos developers are therefore encouraged to use this branch, especially to get a more
interactive testing of the code and some degree of feedback before merging it to the {\bf master}
branch.
This merging will happen on the timescale of months after performing again automated tests and
altering the version tag, e.g. to {\tt 0.x.0}.
The {\bf master} branch is supposed to contain the stable version of Cronos and should be used for
production purposes.
