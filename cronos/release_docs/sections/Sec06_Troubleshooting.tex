\section{Troubleshooting}
\label{sec:trouble_ahead}

List of error messages and possible causes.

\subsection{When compiling}

\begin{itemize}
\item Message:
  \verb+fatal error: +\textit{something}\verb+.h: File not found+ \\
  Execute \verb+locate +\textit{something}\verb+.h+  Execute \verb+locate hdf5.h+
  and add (one of) the found paths as
  \verb+-I+{\em path} behind the \verb+X_INC+ line of file
  \verb+CronosNumLib/makeinclude/include.mak+, e.g. \\
  \verb+-I/usr/include/hdf5/openmpi+ in the case of \verb+hdf5.h+ on the TP4 cluster. \\
  Alternatively, execute \verb+export CPLUS_INCLUDE_PATH=+\textit{path},
  e.g. \\ \verb+export CPLUS_INCLUDE_PATH=/usr/include/hdf5/openmpi+
  if this path was returned. If compiling succeeds, add the {\tt export} line
  to your {\tt .bashrc} file.
\item Message: \verb+/usr/bin/ld: cannot find -lhdf5+ \\
  The HDF5 library is not found by the linker. Do
  \verb+locate libhdf5.a+ and take note of one of the \textit{path}s in
  the output. Edit the file \verb+makeinclude/include.mak+ by inserting
  {\tt -L}\textit{path} at the end of the line defining \verb+X_LIB+.
  Repeat installation steps.  
\item Message \\
\verb+ ../modules.C:22: error: cannot convert ‘int*’ to ‘ProblemType*’ in assignment+
  Possible cause: You did not \verb+#include+ your new mod file into the header
  of {\tt modules.C} (see Sect.~\ref{sect:modules}).
\end{itemize}

\subsection{At runtime}
\begin{itemize}

\item {\tt Illegal instruction} \\
  Possible cause: Might be related to a known bug in the grid output routine.
  Disable the {\tt WriteGrid} keyword in the cat file, then try again.

\item 
\begin{verbatim}
proj: <home>/include/matrix_3d_inc.H:4: T&
Matrix<T,rank>::operator()(int, int, int) [with T = double,
int rank = 3]: rank> Assertion `(i >= lo[0]) && (i <= hi[0])'
failed.
\end{verbatim}
  Possible cause: Trying to access a matrix element outside the matrix'
  bounds. Check indices, remember different extension of {\bf A} and
  {\bf B} fields. Also comes up when trying to access
  \verb+om[qB{x,y,z}]+ with {\tt FLUID\_TYPE} set to {\tt CRONOS\_HYDRO}.

\item Message: {\tt env:}{\em datadir}{\tt:\ Permission denied} \\
  Possible cause:
  If the message includes \\
  {\tt Unrecognized argument} {\em machinedir}{\tt /proj\_MPI ignored.}\\
  {\tt Program binary is:} {\em datadir}\\
  then your {\tt  proj\_MPI} executable might be missing. Recompile using
  {\tt make proj\_MPI}.
\item Message: {\tt Unable to open a GM port} \\
  Possible cause:
  You are already running the maximum number of clients on a machine,
  possibly including leftover orphaned processes after a non-clean
  exit.  Kill all unwanted processes and retry.

\item Message: {\tt Unrecognized argument} ... {\tt ignored.} \\
  Possible cause: Wrong ordering of arguments. Use \\
  \verb+<mpirun> -np+ $n$ \verb+-machinefile+ machinefile
  {\em machinename}\verb+/proj_MPI+ {\em poub} {\em pname} \ .
  
%\item A mismatch of HDF libraries (e.g.\ 1.8.4 vs.\ 1.8.5) is reported.\\
%  Possible cause:
%  Your compiled libraries are not found, and system-installed libraries are
%  substituted, which have a different version number. Make sure the directory
%  path is spelled correctly. Also, adding
%\begin{verbatim}
%LD_LIBRARY_PATH=${LD_LIBRARY_PATH}:~/lib/Linux-amd64/
%export LD_LIBRARY_PATH
%\end{verbatim}
%to your \verb+~/.bashrc+ may help. If all else fails, consider installing
%the required version of HDF5 (rather than the one recommended by this manual)
%and recompile.
\item An error message similar to \\
  {\tt
    Linux-amd64/proj\_MPI: error while loading shared libraries:
    libhdf5.so.6: cannot open shared object file: No such file
    or directory } is issued. \\
  Possible fix:
  Find path of library via \verb+locate libhdf5.so.6+, then issue \\
  \verb+declare -x LD_LIBRARY_PATH=+\textit{path-to-library} .
\end{itemize}


\subsection{git-related}

\begin{itemize}
\item You messed with a file \textit{myfile}, and want to undo your changes
  by reverting to the repository version of that file. This is achieved by
  typing \mbox{\tt git reset --hard origin/master} in that directory.
\item A push is rejected because the repository file has changed since you
  last pulled it. To discard(!) your changes and revert to the repository
  version, first fetch all changes using {\tt git fetch --all}, then reset the
  master using \mbox{\tt git reset --hard origin/master}, and finally
  {\tt pull} again.
\end{itemize}
