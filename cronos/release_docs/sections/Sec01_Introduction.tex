\section{Introduction}

This text gives a brief introduction to the numerical 3D-MHD
simulation code {\sc Cronos}. For a peer-reviewed reference paper see
\citet{L:Kissmann_EA}.
The code is written in C++ and maintained by Ralf Kissmann
{\tt <ralf.kissmann@uibk.ac.at>}. Just like this document, it is in
a state of continuous modification, and comes with no warranties
whatsoever.

\subsection{Disclaimer}

Both the code itself and this manual are not to be distributed without
the consent of the respective authors, except to colleagues working on
the same project for which this consent has already been granted.

\subsection{Requirements}

The code has so far only been used on Linux and Linux-like
architectures (both 32 and 64bit). Root privileges are {\it not}
required to compile and run. For simulations on multiple cores, a
working installation of MPI (e.g.\ {\em OpenMPI} or {\em mpich}) is
required.

\subsection{Notable features}

{\sc Cronos} employs a semi-discrete finite-volume scheme to integrate
the equations of single- or multifluid magnetohydrodynamics forward in time.
For the magnetic field, a constrained transport scheme is used to keep the
field divergence-free. The global timestep is adaptively chosen such
that a pre-defined CFL number is not exceeded. The code can run in
parallel on multiple kernels/CPUs on a single machine, as well as on
multiple machines over a network connection. Binary output files are
generated using HDF5 (hierarchical data format, see
\url{https://www.hdfgroup.org/downloads/hdf5}). Currently supported coordinate systems are
Cartesian, cylindrical, and spherical. The grid can be deformed in
a user-definable fashion, provided that grid lines remain orthogonal. \\
The handling of cylindrical and spherical coordinate singularities for MHD
runs is included except for the spherical origin ($r=0$). For pure hydro runs,
the latter may be implemented by the inclined user, while for MHD this is not
possible because it would require modifications too deep inside the main code.


\subsection{Grid layout and coordinates}
\label{grid+coords}

{\sc Cronos} uses a logically rectangular, three-dimensional grid. The
cell indices $i,j,k$ run from 0 to $N_{x,y,z}-1$, such that an
$[N_x,N_y,N_z]$ grid comprises $(N_x \times N_y \times N_z)$ cells.
For constant spacing, this implies a mapping
\begin{eqnarray}
\label{xi_lin}
x_i = x_0 + i \ \Delta x \\
y_j = y_0 + j \ \Delta y \\
\label{zi_lin}
z_k = z_0 + k \ \Delta z
\end{eqnarray}
from index space to coordinate space, where integer indices designate
the cells' centers. This implies that the simulation actually covers
the $x$ interval
\[ [x_{-1/2}, x_{N-1/2}] = [x_0-\Delta x/2,x_{N-1}+\Delta x/2] \] of length
$(x_{N-1}-x_0)+\Delta x = N \ \Delta x$ in physical space (and similarly
for $y$ and $z$). For reasons of backward compatibility, it is also
possible (though not recommended) to clip cells containing either $0$ or
$N-1$ in their indices, such that the covered space only extends over
\mbox{$x \in [x_0, x_{N-1}]$} (with $N-2$ full cells framed by two half-cells
in each direction, such that edge cells only have a 1/4 of their usual
volume, and corner cells a mere $1/8$).\\
At present, Cartesian, cylindrical, and spherical coordinates
are supported, with user-definable grid spacing (i.e.\ variable $\Delta$'s)
along coordinate lines.\\
Fluid variables ($\rho, u_x, u_y, u_z, e, p$) are localized at the
cell centers ${\bf r}_{i,j,k}$, while the magnetic field
components are face-centered on (and normal to) the respective cell
faces. The vector potential is centered on (and pointing along)
the cell edges (see Table~\ref{tab:ab_comp}).\\
Note that, since C++ syntax requires array elements to have integer indices,
the internal index is offset from the physical one by $-1/2$ for the magnetic
field and by $+1/2$ for the vector potential. This means that a shifted index
$\alpha \in \{x,y,z\}$ runs from $-1$ (rather than 0) to $N_{\alpha}-1$ for
{\bf B} and from 0 to $N_{\alpha}$ for {\bf A}, yielding a total of
$N_{\alpha}+1$ values in both cases\footnote{This unfortunate inconsistency
  merely exists for historical reasons, and will hopefully be corrected in a
future revision.}.

\begin{table}[!hb]
  \begin{center}
    \begin{tabular}{c|c|c|c}
      vector component & physical location & index notation & range \\
      \hline
      $v_x$ & ${\bf r}_{i,j,k}$ & {\tt [i,j,k]} & \\
      $v_y$ & ${\bf r}_{i,j,k}$ & {\tt [i,j,k]} & $[0,m]$\\
      $v_z$ & ${\bf r}_{i,j,k}$ & {\tt [i,j,k]} & \\
      \hline
      $B_x$ & ${\bf r}_{i\pm 1/2,j,k}$ & {\tt [i(-1),j,k]} & \\
      $B_y$ & ${\bf r}_{i,j\pm 1/2,k}$ & {\tt [i,j(-1),k]} & $[0(-1),m]$\\
      $B_z$ & ${\bf r}_{i,j,k\pm 1/2}$ & {\tt [i,j,k(-1)]} & \\
      \hline
      $A_x$ & ${\bf r}_{i,j\pm 1/2,k\pm 1/2}$ & {\tt [i,j(+1),k(+1)]} & \\
      $A_y$ & ${\bf r}_{i\pm 1/2,j,k\pm 1/2}$ & {\tt [i(+1),j,k(+1)]} &
      $[0,m(+1)]$ \\
      $A_z$ & ${\bf r}_{i\pm 1/2,j\pm 1/2,k}$ & {\tt [i(+1),j(+1),k]} & \\
      \hline
    \end{tabular}
    \caption{ \label{tab:ab_comp}
      Localization of {\bf A} and {\bf B} components, using $m:= N-1$.}
  \end{center}
\end{table}

\subsection{Equations}

In its most basic set-up, {\sc Cronos} numerically solves the
equations of ideal magnetohydrodynamics 
\begin{eqnarray}
  \label{eq:cont}
  \partial_t \rho + \nabla \cdot \left( \rho {\bf u} \right) &=& 0 \\
  \partial_t \left( \rho {\bf u} \right)
  + \nabla \cdot \left[ \rho {\bf u u} 
    + (p + \|{\bf B}\|^2/2 ) {\cal I} - {\bf B B} \right] &=& {\bf 0} \\
  \label{eq:NRG}
  \partial_t e + \nabla \cdot \left[(e+p+\|{\bf B}\|^2/2) {\bf u}
    - ({\bf u} \cdot {\bf B}) {\bf B} \right] &=& 0 \\
 \partial_t {\bf B} + \nabla \times {\bf E} &=& {\bf 0} \\
\end{eqnarray}
subject to the closure relations
\begin{eqnarray}
  e &=& \frac{p}{\gamma-1} + \frac{\rho \|{\bf u}\|^2 }{2}
  + \frac{\|{\bf B}\|^2}{2} \\
  \label{eq:ohm}
  {\bf E} + {\bf u} \times {\bf B} &=& {\bf 0} \quad \mbox{and} \\
  \nabla \cdot {\bf B} &=& 0 \ .
\end{eqnarray}
Here, $\rho$, $p$, and $e$ are scalar functions of space ${\bf r}$ and
time $t$, and both {\bf u} and {\bf B} are three-dimensional vector
fields (which also depend on {\bf r} and $t$). ${\cal I}$ is the unit
tensor. Alternatively, the energy equation (\ref{eq:NRG}) can be
replaced by an isothermal or adiabatic equation of state (see
Section~\ref{sect:NRG}).

If so desired, the set of equations
(\ref{eq:cont}) to (\ref{eq:NRG}) can be augmented by additional
terms, both of divergence form and as source terms on the respective
right hand sides. This also includes the possibility to negate any
terms by adding them again with a minus sign.  In addition, the user
is free to provide an arbitrary number of additional equations for
additional variable fields which are to be solved simultaneously with
the above set.
